is the aim of this years challenge to be able to classify diagnoses both from 2 and 12 lead ECG. This can give benefit for methods such as Holter-ECG and not only the clinically used 12-lead ECG. 
\newline\indent This study aims to develop an algorithm for automatic identification of cardiac abnormalities based on “The China Physiological Signal Challenge 2018”, containing 6877 12-lead ECG recordings, sampled at 500 Hz with a length between 6 and 60 seconds and labeled with 9 classes.


Automated ECG interpretation can assist health personnel in diagnosing ECGs. Current clinically used ECG use rule based algorithms which seems to be poor at detecting arrythmias, conduction disorders and pace maker rythms. On the other hand, modern artificial intelligence (AI) methods have proven to be superior in many fields. In the field of ECG interpretation, it has proven to work generally well in classifying arrhythmias, but there is still a lot work to be done in the field of using AI on all sorts of heart related pathology.