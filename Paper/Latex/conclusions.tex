The aim of the present research was to compare ten models and their feasibility to classify 12-lead ECGs with a large number of different diagnoses from multiple sources around the world. The findings reported, shed new light on the results reported in \cite{singstad_convolutional_nodate} by scoring all models using 10-folded cross-validation. In addition, two new ensemble models were developed. The one that used ECG features from all 12 leads outperformed all other models in this study. Also, the ensemble model, using only features from 2 leads, showed promising results compared to the 12-lead models. Although this study focuses mainly on 12-lead ECG, these findings may well have a bearing on Holter-ECG which normally utilizes one or two leads.

The second aim of this study was to investigate the usefulness of LIME for explaining the prediction from a CNN model and an ensemble model. The results of this investigation showed two different ways of explaining the prediction by the ECG classifiers. Despite of its limitations, the study certainly adds to our understanding of how explainable AI can be used for ECG classification. A new study with a greater focus on explainability could produce findings that may be of interest to doctors and cardiologists.