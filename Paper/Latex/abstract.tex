
Existing, commercially available ECG algorithms are mainly rule-based and have limitations in their accuracy. Machine learning, which has shown great performance in many fields over the last years, can possibly outperform the existing ECG classification algorithms. This study is based on the Physionet/Computing in Cardiology challenge 2020 which aimed to classify multiple diagnoses based on $43101$ 12-lead ECGs. The models presented are convolutional neural networks and ensemble models build by clustering and random forest algorithms. These models are complex and often seen as black boxes in terms of explainability. This study addresses this problem by showing how local interpretable model-agnostic explanations (LIME) can be implemented to possibly explain the predictions of complex models.

The best Ensemble model, utilizing features from all 12 leads, outperformed the convolutional neural networks in this study, with an average cross-validated Physionet/Computing in Cardiology Challenge score of $0.512\pm 0.006$. This score is only $0.021$ behind the cross-validated score, on the same development set, reported by the winner of the Physionet/Computing in Cardiology Challenge 2020.

